\documentclass[a4paper,11pt]{article}

\usepackage[utf8]{inputenc}
\usepackage[french]{babel}
\usepackage[margin=3cm]{geometry}
\usepackage{setspace}
\usepackage{aeguill}

\setlength{\parskip}{0mm}
\linespread{1.5}


\begin{document}

\centerline{Frezza-Buet \hfill \today}
\centerline{Romain \hfill}
\centerline{TS3 \hfill}

\centerline{Philosophie}
\centerline{\sc \large Commentaire du texte de ALAIN}
\vspace{1cm}

\hrule
\vspace{4cm}
\hrule
\vspace{1cm}

Toutes les informations que l'on perçoit à l'aide de nos cinq sens constituent notre réalité. Alain se demande si le monde qui l'entoure est une réalité vraie ou fausse par manque d'informations perçues. Dans ce texte il explique qu'il faut penser, réfléchir car la perception du monde que nous avons est incomplète. Nos cinq sens sont limités et ne nous permettent donc pas de capter la totalité des informations de ce que l'on perçoit. Il pense qu'il faut alors douter de notre réalité. Il commence par donner sa définition de \og penser \fg\ (l 1) puis il fait une comparaison (l 1-3). Il se pose alors des questions (l 3-4) pour enfin formuler de façon générale sa thèse (l 4-7) qu'il explique ensuite (l 7-9). Après il donne des exemples (l 9-13) puis finit enfin par reformuler la définition établie au début (l 13).\\

On remarque que Alain donne dans la première phrase la définition de \og penser \fg\ comme étant l'acte de dire \og non \fg. On peut comprendre que \og penser \fg\ c'est refuser de croire en quelque chose, réfuter afin de mieux réfléchir.

Il fait ensuite une comparaison où le signe du \og oui \fg, gestuelle de la tête sur un axe vertical qui permet de marquer son approbation, est comparé à l'endormissement. Au moment où la tête se relâche lorsque l'on s'assoupit. Dire \og oui \fg\ c'est se reposer, ne plus réfléchir. En revanche le signe du  \og non \fg, mouvement horizontal de la tête qui permet de marquer son désaccord, est comparé au réveil, comme lorsqu'on se secoue la tête pour se remettre les idées en place. On comprend que le fait de dire non est pour lui un appel à la réflexion.

Alain pose alors des questions rhétoriques (l 3-4), il se met à la place du lecteur. Aller à l'encontre du monde ou d'un tyran (individu disposant d'un pouvoir absolu obtenu de façon illégitime) ou encore d'un prêcheur (individu qui fait des sermons, un sophiste), Alain dit que ces combats ne sont que des illusions.

La thèse de Alain (l 4-7) est que c'est à sa propre pensée qu'il faut dire \og non \fg. Il faut la combattre car c'est le seul combat qui existe réellement. La pensée \og rompt l'heureux acquiescement \fg, cela signifie qu'il ne faut pas admettre que ce que l'on perçoit est vrai. Ce sont les mineurs de pensée qui disent \og oui \fg, pour être \og heureux \fg\ car ils ne veulent pas affronter le réel problème de la perception.

Il explique (l 7-9) que le monde, ce que l'on perçoit, n'est que tromperies et illusions car on ne perçoit pas toutes les informations, nous sommes dans un \og brouillard \fg\ qui nous aveugle. Tout cela est dû au fait que l'on ne cherche pas autre chose, une autre alternative à notre existence, à la place que nous, notre pensée, avons dans cette perception.

Alain donne comme exemple le tyran auquel le mineur de pensée est soumis par le seul fait qu'il ne se pose pas de questions. On ne peut se révolter que si l'on essaie de comprendre. De plus, à  force de dire \og oui \fg\ à tout, de tout croire, on se retrouve à douter de tout et à ne plus savoir ce que l'on croit, et par conséquent quelque chose de vrai pourrait \og tomber à faux \fg. Quand Alain parle d'\og esclaves \fg, on pense à \og La République \fg\ de Platon avec l'histoire de la caverne, qui défend cette même thèse.

La dernière phrase est une reformulation de la définition qu'il donne au début. Il oppose la réflexion à la croyance car elle n'est pas certaine, elle est un dogme auquel il ne faut pas se fier.\\

Pour conclure, Alain pense qu'il faut aller à l'encontre de notre pensée car celle-ci est trompée par ce que l'on perçoit. Cependant il est alors possible de sombrer dans le scepticisme si on doit être amené à douter de tout ce que l'on croit. C'est pour une raison comme celle-ci que sa thèse n'est pas pertinente. Cette thèse n'est que théorique car si notre pensée était effectivement trompée, il nous serait impossible de le démontrer car nous sommes tous définis par notre pensée.

Honoré de Balzac a dit \og L'illusion est une foi démesurée \fg.

\end{document}
